\chapter{Gestione dei dispositivi di IO}
I dospositivi di input/output (o I/O), sono tutti quei dispositivi che si connettono alla classica architettura composta solo da processore e memoria centrale. Tra tali dispositivi rientrano: Memorie di massa (HDD e SSD), mouse, tastiera, sensori ecc.
Data l'eterogeneità di tali periferiche è richiesto che queste ultime siano gestite in un certo modo, o almeno, che la loro gestione principale sia di un certo tipo. Ciò, quindi, pone le basi su come dovremmo interfacciarci all'utilizzo di tali dispositivi

\section{Architettura generale di un dispositivo di I/O}
In generale un dispositivo di I/O può essere visto come l'insieme di tre parti fondamentali:
\begin{itemize}

    \item \textbf{Registri Dato, Stato e Controllo}: Tali registri sono quelli che interagiscono in maniera diretta con la CPU, e vengono utilizzati da quest'ultima per controllare e gestire le informazioni di quel dato dispositivo. Tali registri sono presenti internamente all'architattura del Calcolatore (ad esempio sulla scheda madre)
    
    \item \textbf{Sistema di adattamento}: Il sistema di adattamento adatta i segnali provenienti dal mondo esterno per essere letti o scritti nei registri di Dato, Stato e Controllo, e quindi permette di adattare l'attacco esterno (tipo l'USB che utilizza comunicazioni sequenziali), con la comunicazione parallela che il processore ha con i registri
    
    \item \textbf{Mondo esterno}: Per mondo esterno si intende tutta la parte che interagisce con il dispositivo in maniera fisica, ed il dispositivo fisico stesso. Quindi immaginiamoci anche una tastiera con il suo connettore USB

\end{itemize}

Un esempio di dispositivo esterno è la memoria HDD.
La memoria HDD ha difatti i tre registri di Dato, Stato e Controllo, quando si vuole scrivere su tale memoria, la CPU va a modificare i registri in modo da garantire tale operazione. Mentre la CPU modifica tali dati, il sistema di adattamento converte i dati presenti in quei tre registri in movimenti della testina + scrittura, rispettando sempre i controlli dati dalla CPU. La scrittura/lettura dei dati tramite la testina e la testina stessa rappresentano, invece, il mondo esterno.
Un altro esempio di periferica è la classica porta UART, che trasmette i sui dati in serie, ma il suo controllo avviene in parallelo. Pertanto al suo interno avrà sia un timer per scandire il clock in base alla tipologia di comunicazione, e poi avrà un buffer parallelo serie, che converte l'informazione da trasmettere in tanti bit seriali. Oltre alla parte parallelo-serie sarà anche dotato di una parte serie-parallelo, nel caso della ricezione.


\subsection{Modalità di comunicazione}
Le tipologie di collegamento che si possono avere tra un processore e le sue periferiche sono le seguenti:
\begin{itemize}
    \item \textbf{Collegamento passivo}: la periferica e la CPU non condividono alcun tipo di comunicazione. Quindi la CPU presuppone che la periferica sia sempre pronta ed è quindi solo lei a decidere quando e come utilizzare i dati, anche se questi magari non sono pronti o ben processati
    \item \textbf{Collegamento Sincrono}: La periferica e la CPU comunicano tra loro, la comunicazione è sincronizzata da un clock comune
    \item \textbf{Collegamento con Handshacking}: L'handshackin è una modalità di sincronizzazione asincrona, poichè si sfruttano dei segnali di comunicazione tra la CPU e la periferica che permettono di capire quando il dato è "pronto" o meno. Una classica implementazione è quella del segnale di req che viene alzato dal processore per far capire che vuole leggere e dall'ack emanato dalla periferica che fa comprendere che il dato è pronto o che è stata presa in carico l'operazione
    \item \textbf{Collegamento semisincrono}: Si condividono le stesse modalità di una comunicazione con handshacking, con la differenza che la sincronizzazione delle due parti avviene mediante uno stesso clock
\end{itemize}

\subsection{Interfacciamento CPU e periferica}
Per utilizzare le periferiche la CPU deve poter accedere ai registri di Dato, Stato e Controllo di tali periferiche. Le tipologie di interfacciamento che ci possono essere tra CPU e Periferica sono:
\begin{itemize}
    \item \textbf{Memory Mapped I/O}: La CPU fa riferimento ai registri di Dato, Stato e Controllo di una periferica come se fossero dei registri in memoria
    \item \textbf{I/O Mapped}: La CPU ha specifici comandi per interagire con le periferiche di I/O
\end{itemize}

Nel nostro caso il Motorola 68k è una tipologia di architettura memory mapped, e quindi la trattazione dei registri avviene mediante i classici comandi di spostamento già utilizzati

\subsubsection{Memory Mapped I/O}
Nel caso di interfacciamento con una struttura Memory Mapped, l'accesso ai registri di una determinata periferica avvengono tramite i bus di collegamento classici, che collegano anche la memoria ecc. Ciò quindi mi limita nell'utilizzo degli indirizzi, poichè, quando faccio riferimento ad un registro di una periferica, tale indirizzo non deve appartenere al set di indirizzi della memoria centrale

\subsubsection{I/O Mapped}
Nel caso di interfacciamento con una struttura I/O Mapped, l'accesso ai registri di una determinata periferica avviene mediante degli specifici comandi. Questo perchè le periferiche sono collegate a bus dedicati o hanno una gestione dedicata, che quindi differisce dalle comunicazioni che avvengono in generale all'interno dell'architettura al costo di avere meno modi di indirizzamento, dato che non si userà più la MOVE che è un codice operativo ortogonale

\subsubsection{Logiche di selezione}
Quando devo selezionare la mia periferica a cui faccio riferimento, utilizzo una serie di indirizzi. Tali indirizzi possono essere utili al fine di realizzare i seguenti tipi di logica:
\begin{itemize}
    \item \textbf{Logica tristate}: Logica che quando una periferica non vede il suo indirizzo sui bus adeguati smette di interagire con il sistema, quindi ignora la variazione dei dati sul bus. Tale logica, quindi utilizza l'indirizzo interno della nostra periferica
    \item \textbf{Logica Plug-and-play}: L'indirizzo della periferica viene scelto in base ad una serie di indirizzi disponibili
\end{itemize}

\subsection{BUS}
I bus sono i collegamenti che interconnettono le varie componenti di un calcolatore, ovvero, CPU, memoria e periferiche di I/O.
In generale non vi è una tipologia unica di bus, ve ne sono varie in base alla tipologia di utilizzi e alla tipologia di tecnologie utilizzate.
I bus, si contraddistinguono principalmente per la divisione che attuano sui loro collegamenti, ma in generale, le informazioni che vengono trasportate sono solitamente le stesse.
Le informazioni, quindi, sono dipartite tra i vari collegamenti presenti in un BUS. I collegamenti generici che si possono identificare in un bus sono:
\begin{itemize}
    \item \textbf{Alimentazione}: Collegamenti che principalmente comprendono la VCC (o più VCC), che sarebbero le tensioni di alimentazione delle componenti; ed il cavo di terra (o GND)
    \item \textbf{Dati}: Collegamenti che trasportano i dati che vengono interscambiati tra i vari dispositivi
    \item \textbf{Indirizzo}: Collegamenti che trasportano gli indirizzi che permettono la selezione dei dispositivi interessati o dei registri a cui si vuole accedere
    \item \textbf{Controllo}: Collegamenti che trasportano le informazioni inerenti alla tipologia di operazione che si vuole effettuare
    \item \textbf{Stato}: Collegamenti che permettono il controllo di flusso e la segnalazione di eventuali conflitti o errori
\end{itemize}

Data una tipologia di bus, può capitare che la periferica che vado ad utilizzare non è ad-hoc per quella determinata tipologia di bus. Pertanto, quello che posso fare, è considerare l'utilizzo di un \textbf{adapter}, che mi permette di adattare il bus classico con la tipologia di attacco specifica per la mia periferica. Oltretutto in alcuni casi, quando il dispositivo non permette la configurazione degli indirizzi, per evitare conflitti, l'adapter gestisce anche la gestione di tale indirizzo rispetto al sistema

\subsection{Driver}
I driver sono dei programmi che permettono di capire come il processore vada ad utilizzare una determinata periferica.
Le tipologie di approccio che si possono avere nella scrittura dei driver sono varie, la più primitiva è il polling.
Il \textbf{Polling} è un modo con cui il processore va ad interagire con la periferica. In generale si va a dare un primo segnale di controllo alla periferica e si aspetta uno specifico valore di stato per poter accedere al dato. Tali sistema è altamente inefficiente, poichè mentre la CPU aspetta la riposta della periferica passano dei periodi di clock dove la CPU rimane ferma. Il tempo che quindi la CPU rimane senza eseguire delle operazioni utili è detto \textbf{Busy-waiting}. Un possibile codice di implementazione del polling è [\ref{m68:polling}]

\begin{lstlisting}[caption={Codice polling}, label=m68:polling]
        ORG      $8000
*Inizializzo lo stato dei miei registri
        MOVE.B   #$00,C
        MOVE.B   #$00,S
*Vado a considerare la zona di memoria dove voglio salvare i dati
        MOVEA.L  #VAR1,A0
        MOVE.W  #0,D0
*Devo prelevare N dati quindi ciclo N volte
FOR     CMP.W   #N,D0
        BGE   FUORI

*Qui devo scrivere il driver sapendo  che devo ricevere un byte

        MOVE.B  #$01,C  *Vado a settare un controllo
*Qui inizia il ciclo di polling dove attendo uno specifico valore dello stato
L1      MOVE.B  S,D1
        AND.B  #$80,D1  *Se il bit si e' alzato ho finito. Altrimenti continuo ad aspettare
        BEQ  L1  

*Qui il dato e' stato letto, poiche' ho il flag di stato alzato
        MOVE.B  D,(A0)+ *Inserisco il dato in memoria
        MOVE.B  #$00,C  *Vado a resettare il segnale di Controllo
        MOVE.B  #$00,S  *Vado ad "eludere" il sistema su un segnale di stato

FUORI   ADD.B  #1,D0             *Incremento il conteggio
        BRA  FOR                 *Ripeti
        
        ORG     $8100
D       DS.B    1   *Registro dato
S       DS.B    1   *Registro Stato
C       DS.B    1   *Registro Controllo

N       EQU     5   *Quantita' di valori da considerare
VAR1    DS.B    5   *Array effettivo di raccolta dati
\end{lstlisting}
\newpage
Il codice [\ref{m68:polling}] presenta però le seguenti criticità:
\begin{itemize}
    \item \textbf{Mancata Generalizzazione}: Si vanno a considerare in maniera diretta i registri in memoria D,S e C. Che per l'implmentazione di un driver riutilizzabile non è proprio la scelta corretta
    \item \textbf{Polling}: L'attesa che viene svolta all'interno di tale codice non permette al processore di eseguire altri passi prima di aver ricevuto tutti i caratteri
    \item \textbf{Gestione dei malfunzionamenti}: Se la periferica ha un qualunque tipo di malfunzionamento e quindi non aggiorna mai il registro di stato, tale ciclo eseguirà all'infinito senza mai fermarsi
\end{itemize}

Le due problematiche (o criticità), possono essere affrontate in vario modo. Per la prima la soluzione è molto semplice, al posto di andare a considerare i registri di Dato, Stato e Controllo in maniera diretta, possono essere considerati come registri indirizzo (Ai), a cui vado ad associare gli indirizzi di tali registri. Tali inidirizzi poi vengono settati secondo un determinato criterio prima della chiamata al driver.
Per ovviare, invece, al secondo problema c'è il bisogno di considerare le \textbf{interruzioni}. Mentre per l'ultimo problema la soluzione è l'introduzione di \textbf{timer}, che permettono di capire quando un sistema sta impiegando un tempo più grande del dovuto per eseguire un operazione, ciò permette di poter gestire ed uscire da situazioni di eventuali guasti.

\subsubsection{Interruzioni}
L'interruzione è un evento che cambia la normale esecuzione di un programma per fargli eseguire prima del codice specifico per la gestione di quella determinata condizione [\ref{img:bootint}]. In generale non è corretto parlare solo di interruzioni, poichè tale termine non comprende o non può comprendere anche il caso in cui le interruzioni vengano scatenate dall'interno per casistiche particolari. Difatti è più corretto fare la seguente suddivisione:
\begin{itemize}
    \item \textbf{Interruzioni}: Segnali che sono a contatto con le periferiche e che permettono alla CPU di interrompersi e di eseguire il codice per la gestione della comunicazione con quella data interfaccia. Le interruzioni sono scatenate, quindi, dal dispositivo che vuole interagire con la CPU
    \item \textbf{Eccezioni}: Funzionano come le interruzioni, con la differenza che vengono scatenate internamente rispetto al processore, quindi non vengono gestite dai dispositivi ma dal programma stesso, tale condizione fa eseguire comunque una ISR, con l'obbiettivo di dover gestire particolari casistiche (es. divisione per 0)
\end{itemize}

\begin{figure}
    \centering
    \includegraphics[width=0.5\textwidth]{img/BootInt.png}
    \caption{Ciclo di esecuizone con interrupt}\label{img:bootint}
\end{figure}
Quindi quando le interruzioni sono scatenate vanno ad effettuare una chiamata a subroutine particolare, tale chiamata è detta ISR (Interrupt-Services-Routine). Tale situazione, quindi, ferma il sistema dalla sua normale esecuzione del programma per dare priorità alla gestione dell'interruzione. Questo, quindi, apre molti dubbi su come gestire lo stato in cui si trova la macchina, poichè se quando torno dalla ISR, ho cambiato qualche registro significativo si potrebbe compromettere il normale funzionamento del programma. In generale i due registri che richiedono l'obbligo di essere salvari sono i registri: \textbf{SR(Status register)} e il \textbf{PC(Program Counter)}. In generale, i registri che vado a salvare in questo passaggio sono anche detti: \textbf{Descrittore di processo}, tali registri, quindi, descrivono lo stato di funzionamento del mio processore quando poi è stato prelazionato dalla mia ISR. Ciò mi permette di proseguire ancora con la normale esecuzione del programma prefissato. 

\subsubsection{Gestione delle Interruzioni}
Una volta definito cosa sono le interruzioni è di fondamentale importanza capire come il processore le gestisce. Le principali modalità di gestione delle interruzioni sono due e sono:
\begin{itemize}
    \item \textbf{Vettorizzate}: Ogni livello di priorità di interrupt è collegato al processore. I fili di collegamento per le interrupt sono limitati, quindi più dispositivi possono collegarsi sullo stesso cavo di interrupt. Il processore, quindi, per identificare il dispositivo che ha scatenato l'Interrupt va a controllare i bus, su cui il dispositivo ha caricato il suo codice identificativo. Identificare il dispositivo, vuol dire identificare la tipologia di ISR da andare ad utilizzare. Gli indirizzi degli entry-point delle varie periferiche sono memorizzati in memoria a partire dall'indirizzo 0 a seguire per 256 locazioni di 4 byte. Tali locazioni si dividono nel seguente modo:
    \begin{itemize}
        \item \textbf{Funzioni speciali}: Da 0 a 24, gli entry-point identificano delle funzioni speciali o di gestione aritmentica
        \item \textbf{Interruzioni autovettorizzate}: da 25 a 31 sono indicizzate le locazioni per il funzionamento autovettorizzato
        \item \textbf{Trap}: da 32 a 47 sono indicizzate le funzioni per la gestione dei Trap
        \item \textbf{Utilizzabili}: da 48 a 256 sono locazioni disponibili per l'inserimento degli entry-point per la gestione di diverse periferiche
    \end{itemize}
    \item \textbf{Autovattorizzate}: A differenza del caso vettorizzato, evita la lettura del codice identificativo, poichè ogni livello di interrupt è collegato al vettore delle ISR autovettorizzate e permette di selezionare in maniera "ignorante" l'ISR alla locazione della tipologia di priorità inserita
\end{itemize}

In generale, nel caso di sistema \textbf{vettorizzato}, viene in aiuto il componente \textbf{PIC (Programmable Interrupt Controller)} [\ref{img:PIC}]. Che è un dispositivo che accetta in ingresso 8 fili e kfa uscire l'indirizzo dell'entry point per la chiamata dell'ISR. Ciò permette di sollevare il processore dalla gestione diretta degli interrupt e quindi l'utilizzo di un solo filo di interruzione + la lettura dell'entry-point dell'ISR in maniera diretta. Il PIC gestisce anche tutto il sistema di priorità in maniera ad-hoc.
Oltre al PIC ci sono anche altri sistemi che permettono di sollevare il processore dalla gestione delle periferiche. Un esempio è il \textbf{DMA}, che gestisce il passaggio dei dati periferica-memoria, senza dover passare per i registri del processore. Esso viene settato ad un valore che va a decrementare mano a mano che riceve i dati dalla periferica.

\begin{figure}
    \centering
    \includegraphics[width=0.7\textwidth]{img/PIC.png}
    \caption{PIC(Programmable Interrupt Controller)}\label{img:PIC}
\end{figure}

